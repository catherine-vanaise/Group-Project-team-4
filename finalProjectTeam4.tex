% Options for packages loaded elsewhere
\PassOptionsToPackage{unicode}{hyperref}
\PassOptionsToPackage{hyphens}{url}
%
\documentclass[
]{article}
\usepackage{lmodern}
\usepackage{amsmath}
\usepackage{ifxetex,ifluatex}
\ifnum 0\ifxetex 1\fi\ifluatex 1\fi=0 % if pdftex
  \usepackage[T1]{fontenc}
  \usepackage[utf8]{inputenc}
  \usepackage{textcomp} % provide euro and other symbols
  \usepackage{amssymb}
\else % if luatex or xetex
  \usepackage{unicode-math}
  \defaultfontfeatures{Scale=MatchLowercase}
  \defaultfontfeatures[\rmfamily]{Ligatures=TeX,Scale=1}
\fi
% Use upquote if available, for straight quotes in verbatim environments
\IfFileExists{upquote.sty}{\usepackage{upquote}}{}
\IfFileExists{microtype.sty}{% use microtype if available
  \usepackage[]{microtype}
  \UseMicrotypeSet[protrusion]{basicmath} % disable protrusion for tt fonts
}{}
\makeatletter
\@ifundefined{KOMAClassName}{% if non-KOMA class
  \IfFileExists{parskip.sty}{%
    \usepackage{parskip}
  }{% else
    \setlength{\parindent}{0pt}
    \setlength{\parskip}{6pt plus 2pt minus 1pt}}
}{% if KOMA class
  \KOMAoptions{parskip=half}}
\makeatother
\usepackage{xcolor}
\IfFileExists{xurl.sty}{\usepackage{xurl}}{} % add URL line breaks if available
\IfFileExists{bookmark.sty}{\usepackage{bookmark}}{\usepackage{hyperref}}
\hypersetup{
  pdftitle={Impact of Government Expenditure in Education on Unemployment},
  pdfauthor={Anna Canestro, Serena Foini \& Catherine Vanaise},
  hidelinks,
  pdfcreator={LaTeX via pandoc}}
\urlstyle{same} % disable monospaced font for URLs
\usepackage[margin=1in]{geometry}
\usepackage{color}
\usepackage{fancyvrb}
\newcommand{\VerbBar}{|}
\newcommand{\VERB}{\Verb[commandchars=\\\{\}]}
\DefineVerbatimEnvironment{Highlighting}{Verbatim}{commandchars=\\\{\}}
% Add ',fontsize=\small' for more characters per line
\usepackage{framed}
\definecolor{shadecolor}{RGB}{248,248,248}
\newenvironment{Shaded}{\begin{snugshade}}{\end{snugshade}}
\newcommand{\AlertTok}[1]{\textcolor[rgb]{0.94,0.16,0.16}{#1}}
\newcommand{\AnnotationTok}[1]{\textcolor[rgb]{0.56,0.35,0.01}{\textbf{\textit{#1}}}}
\newcommand{\AttributeTok}[1]{\textcolor[rgb]{0.77,0.63,0.00}{#1}}
\newcommand{\BaseNTok}[1]{\textcolor[rgb]{0.00,0.00,0.81}{#1}}
\newcommand{\BuiltInTok}[1]{#1}
\newcommand{\CharTok}[1]{\textcolor[rgb]{0.31,0.60,0.02}{#1}}
\newcommand{\CommentTok}[1]{\textcolor[rgb]{0.56,0.35,0.01}{\textit{#1}}}
\newcommand{\CommentVarTok}[1]{\textcolor[rgb]{0.56,0.35,0.01}{\textbf{\textit{#1}}}}
\newcommand{\ConstantTok}[1]{\textcolor[rgb]{0.00,0.00,0.00}{#1}}
\newcommand{\ControlFlowTok}[1]{\textcolor[rgb]{0.13,0.29,0.53}{\textbf{#1}}}
\newcommand{\DataTypeTok}[1]{\textcolor[rgb]{0.13,0.29,0.53}{#1}}
\newcommand{\DecValTok}[1]{\textcolor[rgb]{0.00,0.00,0.81}{#1}}
\newcommand{\DocumentationTok}[1]{\textcolor[rgb]{0.56,0.35,0.01}{\textbf{\textit{#1}}}}
\newcommand{\ErrorTok}[1]{\textcolor[rgb]{0.64,0.00,0.00}{\textbf{#1}}}
\newcommand{\ExtensionTok}[1]{#1}
\newcommand{\FloatTok}[1]{\textcolor[rgb]{0.00,0.00,0.81}{#1}}
\newcommand{\FunctionTok}[1]{\textcolor[rgb]{0.00,0.00,0.00}{#1}}
\newcommand{\ImportTok}[1]{#1}
\newcommand{\InformationTok}[1]{\textcolor[rgb]{0.56,0.35,0.01}{\textbf{\textit{#1}}}}
\newcommand{\KeywordTok}[1]{\textcolor[rgb]{0.13,0.29,0.53}{\textbf{#1}}}
\newcommand{\NormalTok}[1]{#1}
\newcommand{\OperatorTok}[1]{\textcolor[rgb]{0.81,0.36,0.00}{\textbf{#1}}}
\newcommand{\OtherTok}[1]{\textcolor[rgb]{0.56,0.35,0.01}{#1}}
\newcommand{\PreprocessorTok}[1]{\textcolor[rgb]{0.56,0.35,0.01}{\textit{#1}}}
\newcommand{\RegionMarkerTok}[1]{#1}
\newcommand{\SpecialCharTok}[1]{\textcolor[rgb]{0.00,0.00,0.00}{#1}}
\newcommand{\SpecialStringTok}[1]{\textcolor[rgb]{0.31,0.60,0.02}{#1}}
\newcommand{\StringTok}[1]{\textcolor[rgb]{0.31,0.60,0.02}{#1}}
\newcommand{\VariableTok}[1]{\textcolor[rgb]{0.00,0.00,0.00}{#1}}
\newcommand{\VerbatimStringTok}[1]{\textcolor[rgb]{0.31,0.60,0.02}{#1}}
\newcommand{\WarningTok}[1]{\textcolor[rgb]{0.56,0.35,0.01}{\textbf{\textit{#1}}}}
\usepackage{graphicx}
\makeatletter
\def\maxwidth{\ifdim\Gin@nat@width>\linewidth\linewidth\else\Gin@nat@width\fi}
\def\maxheight{\ifdim\Gin@nat@height>\textheight\textheight\else\Gin@nat@height\fi}
\makeatother
% Scale images if necessary, so that they will not overflow the page
% margins by default, and it is still possible to overwrite the defaults
% using explicit options in \includegraphics[width, height, ...]{}
\setkeys{Gin}{width=\maxwidth,height=\maxheight,keepaspectratio}
% Set default figure placement to htbp
\makeatletter
\def\fps@figure{htbp}
\makeatother
\setlength{\emergencystretch}{3em} % prevent overfull lines
\providecommand{\tightlist}{%
  \setlength{\itemsep}{0pt}\setlength{\parskip}{0pt}}
\setcounter{secnumdepth}{-\maxdimen} % remove section numbering
\ifluatex
  \usepackage{selnolig}  % disable illegal ligatures
\fi

\title{Impact of Government Expenditure in Education on Unemployment}
\author{Anna Canestro, Serena Foini \& Catherine Vanaise}
\date{01/04/2021}

\begin{document}
\maketitle

\hypertarget{introduction}{%
\section{Introduction}\label{introduction}}

Unemployment levels are constantly under the spotlight of governments,
as signals of the economic stability of countries and of the
effectiveness of the policies they implement. It is conventionally
believed that the acquisition of skills enhances the possibilities of
getting hired (Grimaccia \& Lima, 2013). In fact, expenditures in
education are particularly relevant for growth because they ensure a
higher attainment and improve workers' skills, thus leading to a more
prepared human capital who should be able to face the present harsh
working environment. The necessity of educated people is generally an
imperative for all countries, both developed and underdeveloped, exactly
because of the expected economic benefits arising from them.

This paper studies the relationship between educational expenditure by
level of schooling and unemployment rate in twenty-six OECD countries
using yearly data from 2013 to 2017 from the OECD.Stat website. The
study investigates the effect of governmental investments as a
percentage of GDP at each of the four different levels of
education---primary, lower- and upper-secondary, and long-cycle
tertiary---on the proportion of people aged 25-64 unemployed in the
countries of the sample. Our aim was to broaden the extant literature,
which does not explore considerably this distinction. Significant
results would allow for suggestions to be made for governments about
where to direct their limited resources to reduce unemployment in the
most efficient manner.

The work is divided in the following sections: first, a literature
review is presented, and the research gap is identified; then we
formulate the hypotheses we aim to test with the models developed in the
methodology section. Finally, the analysis of the data and the
discussion of results are presented, and conclusions are drawn.

The analysis did not show any significant results for the impact of
expenditures at the first three levels of education and found a positive
relationship between long-cycle tertiary education expenditures and
unemployment, suggesting that the more governments spend in Bachelor,
Master and PhD programs, the higher the level of unemployment. This
particular outcome could be justified by the fact that the greater the
outlays in education at these levels, the higher the qualified labour
supply which, if not equally matched by labour demand, could lead to the
upsurge of unemployment.

\hypertarget{literature-review}{%
\section{Literature Review}\label{literature-review}}

\hypertarget{the-relationship-between-education-and-economic-growth}{%
\subsection{The Relationship Between Education and Economic
Growth}\label{the-relationship-between-education-and-economic-growth}}

Extant literature has investigated the positive relationship between
education and economic growth; it is widely accepted that, as
individuals become more educated, they turn into greater citizens and
contribute to the improvement of the general standard of living of
society (Obi \& Obi, 2014). Moreover, this association has been tested
for different countries from developed countries as the United-States
(Jorgenson \& Fraumeni, 1992) to developing countries, as Tanzania and
Zambia (Jung \& Thorbecke, 2001), and India (Chandra, 2010). All these
scholars have been able to prove that education expenditures have a
positive and significant impact on growth.

Consequently, education is one of the most important destinations of
government expenditures in both developed and developing economies as it
will, in the long run, transform their human, economic, cultural and
social dimensions. In their study, Obi \& Obi (2014) highlight that to
promote economic growth, the efficient use of labor and capital requires
highly trained and skilled individuals. Ensuring that people have access
to the best environment in which they can improve and cultivate their
talents is for institutions as imperative as investing in other factors
of production such as finance, natural resources and physical equipment.
Therefore, since citizens acquire their knowledge and develop their
skills through education, it is crucial for governments -- whose aim is
to foster their countries' economic development -- to spend their
savings in the academic system. On this last point, Quan \& Beck (1987)
add that education expenditures contribute to economic progress in two
ways: attracting people to remain in the State and increasing workers'
productivity. The latter, by extension, improves citizens' standard of
living, creates job opportunities and promotes more equitable income
distribution in the country (Mefteh, Bouhajeb, \& Smaoui, 2016).

Interestingly, Baqir, Harun, \& Razani (2017) emphasize in their study
the compounding beneficial effect of spending in education. Indeed,
these investments would not only refine labor's skills today, but they
will generate even more skilled workers in the future, capable to
satisfy the continuously changing labor market's requirements. If a
government decides to deploy a small portion of its savings to
education, dramatic outcomes might occur for the economy, which will
face a shortage of skilled labor and become unable to compete at the
global level.

\hypertarget{the-relationship-between-educational-expenditure-and-unemployment}{%
\subsection{The Relationship Between Educational Expenditure and
Unemployment}\label{the-relationship-between-educational-expenditure-and-unemployment}}

Other than the acknowledged impact that education has on economic growth
and the general relevance of expenditure in education from governments,
it is necessary to examine the long-term impacts on the employment
level, which can be considered one of the main objectives of investing
in education. Economic growth does not emerge directly from the
expenditure in education but, before generating growth, touches various
social aspects, among which we must also consider employment.
Unemployment is nowadays a strong concern for governments, both for
developed and developing countries; while for the first group it is
crucial to keep a competitive position, for the second it is even more
important to create the so called ``human capital'' needed to raise
domestic economy (Šonje et Al., 2018).

However, despite the shared belief at public expenditures are necessary
to increase the quality of education, also taken into account by
Grimaccia \& Lima (2013), there are no homogenous results regarding the
impact of public spending on employment levels.

On one hand, Grimaccia \& Lima (2013), studied the 27 European Union
countries, showing that higher investments in education lead to higher
employment rates. Dachito, Alemu, \& Alemu (2020), who studied the
influence of government expenditure on unemployment in the Ethiopian
context, found a negative correlation not only between expenditure and
unemployment but also between expenditure and unemployment growth,
validating Keynes's theory -- cited in Dachito, Alemu, \& Alemu (2020).
Fedderke \& Luiz (2006) were able to show the beneficial impact on
employment in the long run in South Africa and Zafer, while William, \&
Leslie (2014) did the same for the United-States.

On the other hand, other scholars reached conclusions that seem to be in
contrast with the previous ones. For example, according to Nurudeen \&
Usman (2010), education expenditures negatively impact growth and Ahec
Šonje, Deskar-Škrbić, \& Šonje (2018), analysing the public expenditure
on education per student, showed its inefficiency in affecting
unemployment.

\hypertarget{the-importance-of-efficient-expenditures-in-education}{%
\subsection{The Importance of Efficient Expenditures in
Education}\label{the-importance-of-efficient-expenditures-in-education}}

In the previous sections, the positive effect of education expenditures
on economic mid-term and log-term growth has been highlighted. However,
governments cannot simply invest all their resources on the national
academic systems for two main reasons. First, there are other equally
important priorities that require considerable spending. Second,
sizeable public debts and fiscal deficits impose government to adopt
great fiscal responsibility (Ahec Šonje, Deskar-Škrbić, \& Šonje, 2018).
Therefore, governments need to efficiently allocate public resources to
education. For example, EU regulations expect members to provide
educational services by minimizing the amount of their individual
national savings devoted to this specific sector, so to control public
budgets (Agasisti, 2014).

It might seem that these directives could limit governments from
benefiting from better education systems. Nonetheless, spending more in
education is not always synonym to higher quality; evidence suggests
that a large number of developed countries could actually increase
students' educational level by allocating fewer resources than those
actually dedicated to education (Gimenez, Prior, \& Thieme, 2007),
justifying authorities' decisions.

To conclude, in a world with finite resources, choosing the most
efficient mix and deciding where to invest to maximize the outcomes is a
central concern to many national stakeholders -- i.e., politicians,
taxpayers, managers. Education efficiency should be seen as a complex
circular system in which investments generate opportunities for students
to learn. Given those opportunities, students will be better equipped to
compete in the job market, and they will perform better, influencing
future funding decisions (Flores, 2017).

\hypertarget{research-gap}{%
\subsection{Research Gap}\label{research-gap}}

Despite the large amount of research on the relationship between
educational expenditure and both economic growth and unemployment, a
large gap in the literature is found regarding the division of these
expenditures among the different educational levels. In particular, it
is not investigated whether and how investments in a particular level of
education -- primary, secondary or tertiary -- provide greater benefit
on the economy. Moreover, the literature generally focuses more on
specific-countries rather than cross-country studies, which could be
extremely helpful in the view of resources allocation decisions.

Therefore, the aim of our research is, first of all, to assess whether
the expenditure in education is negatively correlated to unemployment
levels and, subsequently, to better understand on which educational
level governments should direct their investments to reach the best
possible outcome for diminishing unemployment.

\hypertarget{hypothesis-formulation}{%
\section{Hypothesis Formulation}\label{hypothesis-formulation}}

The literature review emphasizes the importance of education in
fostering countries' economic growth and the subsequent relevance of
government investments in the sector. On the basis of the extant
literature and the gap identified, we develop a set of hypotheses in
order to study whether the impact of government expenditures in
education on unemployment varies according to the destination of these
expenditures -- primary, lower and upper secondary, tertiary schooling.

\hypertarget{primary-education}{%
\subsection{Primary Education}\label{primary-education}}

According to the definition of the OECD, primary education is designed
to provide children between 5 and 11 years old a sound basic education
in reading, writing and mathematics and a basic understanding of some
other subjects (OECD).

Investment in primary education depends on the proportion of young males
and females of school age in the total population and is the largest
share of expenditure on educational institutions. This data is justified
by two reasons. First, primary and secondary education are envisaged as
a standard public good. Second, especially at the primary and secondary
levels, education is perceived as a means to equalize and redistribute
social opportunities (Del Boca, Monfardini, \& See, 2018; Hatos, 2014).
Indeed, because Northern European governments invest more in early
education than Southern European ones, cognitive test scores are higher,
and inequality is lower. Finally, in most OECD countries, the enrollment
rates are close to 100\% at the primary levels, highlighting the need
for governments to invest resources in this education level to provide
the necessary basic knowledge to young citizens. This will give them the
possibility to build the foundations of their academic and professional
career, to contribute to the growth of the country. Hence, we define the
following hypothesis:

\emph{Hypothesis 1: Higher government investments in primary education
will lead to lower unemployment rates.}

\hypertarget{lower-secondary-education}{%
\subsection{Lower secondary education}\label{lower-secondary-education}}

Lower secondary education provides the basics of education in a more
specialist way than the primary level. The typical duration is three
years, and, in some countries, it represents the final step of
compulsory education. On average, across OECD countries, less than 3\%
of youth are out of school in primary and lower secondary education. The
fourth Sustainable Development Goal of the UN focuses on the importance
of education and learning to shape a sustainable future, stressing the
necessity to provide investments in education. The lower secondary
education is particularly relevant for its impact on pupils' future
education; according to Michaelowa (2007), there is a positive
relationship between the quality of secondary and tertiary education.

Therefore, ensuring a smooth transition from lower to upper secondary
education is a crucial step that may prevent students from quitting. We
define the following hypothesis:

\emph{Hypothesis 2: Higher government investments in lower secondary
education will lead to lower unemployment rates.}

\hypertarget{upper-secondary-education}{%
\subsection{Upper secondary education}\label{upper-secondary-education}}

According to the definition of the OECD (2020), upper secondary
education provides a higher specialisation than at lower secondary
level. The attainment of this level of education has become a minimum
requirement to be competitive in the modern labor market. Compared to
their higher-educated peers, young people who leave school before the
competition of the upper secondary education not only have lower job
opportunities but tend to have lower social connectedness (OECD, 2019).
As primary education, secondary education is perceived as a standard
public good and for this reason, a fair amount of governments' resources
must be devoted to its development. Clearly, it is important for the
government to verify whether the costs of its investments will be
recovered. Since an upper secondary education offers good prospects of
employability and tends to translate into higher earnings, governments
will be able to receive higher tax income and social contributions.
Therefore, we propose the following hypothesis:

\emph{Hypothesis 3: Higher government investments in upper secondary
education will lead to lower unemployment rates.}

\hypertarget{tertiary-education}{%
\subsection{Tertiary Education}\label{tertiary-education}}

Tertiary education is strictly related to a country's ability to provide
specializing skills to future workers. Tertiary education is composed by
short cycle and long cycle tertiary programs. The former is highly
specific and is usually designed to provide participants with
professional knowledge, skills and competencies to either enter the
labor market directly or to move to long cycle programs, which are under
the observation in this study. Indeed, the latter is composed of the
Bachelor, Master and PhD or equivalent levels, which represent the
greatest levels of education attainable.

The benefits of gaining a tertiary degree are numerous, including better
salaries and employment likelihood (OECD, 2020). OECD average data for
2019 shows that the employment rates of 25-34 years old passes from 61\%
for those below upper secondary education to 78\% for those with upper
secondary education and 85\% for tertiary education (OECD, 2020).
Although among non-OECD countries, China represents a valid example of
the positive effects of investments in education on graduate employment.
Indeed, the local government had strongly expanded its investments in
higher education starting from 1999 and this led to a boost in graduate
employment levels (Bai, 2006). Therefore, we propose the following
hypothesis:

\emph{Hypothesis 4: Higher government investments in long-cycle tertiary
education will lead to lower unemployment rates.}

\hypertarget{methodology}{%
\section{Methodology}\label{methodology}}

We tested the hypotheses using panel data gathered from the OECD. Stat,
which reports data from the 38 members of the homonym organization.
However, we included only 26 countries in our sample as 12 of the
countries were missing too much information to be properly analyzed.
This samples allows for a cross country analysis. In more detail, the
sample is constituted of Australia, Austria, Belgium, Chile, Czech
Republic, Estonia, France, Germany, Greece, Hungary, Iceland, Ireland,
Israel, Italy, Latvia, Lithuania, Luxemburg, the Netherlands, New
Zealand, Poland, Slovak Republic, Slovenia, Spain, Sweden, the
United-Kingdom and the United-States. Despite the sample of OECD
countries being downsized, we were careful to maintain geographical
heterogeneity.\\
We decided to limit the scope of our analysis to the period from 2013 to
2017 since they were the five most recent years available.

All data were collected from the OECD database to ensure coherent
measurements, being from the same organization. For detailed
explanations demonstrating paths and filters to obtain data, see
\emph{Appendix (X)}.

\hypertarget{dependent-variable}{%
\subsection{Dependent variable}\label{dependent-variable}}

Our dependent variable is the level of unemployment. It is measured as
the percentage of the population aged between 24 and 64 years of age
that is without employment.

\hypertarget{independent-variables}{%
\subsection{Independent variables}\label{independent-variables}}

We consider four independent variables, whose impact on the level on
unemployment is tested in the models. The first one is the level of
total expenditure in primary education (ISCED2011 level 1); the second
one is the level of total expenditure in lower secondary education
(ISCED2011 level 2); the third variable is the level of total
expenditure in upper secondary education (ISCED2011 level 3); the fourth
and last one is the level of total expenditure in long cycle tertiary
education (ISCED2011 levels 6 to 8). They are all included in the model
as percentage of the GDP of the specific country in order to normalize
with respect to the economic development of the different nations. They
individually measure the amount of USD invested by all public and
private institutions in their specific grade of schooling -- primary,
lower and upper secondary, bachelor, master and PhD respectively -- as
percentage of the total gross domestic product. Since the values at the
numerator are reported in US dollars at purchasing power parity as the
denominator, we obtained coherent percentage values. We expect to find a
positive relationship between these independent variables and
unemployment rates.

\hypertarget{control-variables}{%
\subsection{Control Variables}\label{control-variables}}

Based on the literature, factors other than our four levels of education
expenditures might have an impact on unemployment rates and it is
necessary to control for them to avoid any potential bias of our
estimations. The educational attainment measures the percentage of
university graduates in the population aged 25-64 and controls for the
social recognition of education: it reflects the importance of education
for the local population - i.e., the higher the percentage of people
with tertiary education, the greater the recognition of this attainment
in the social and labor environment. Finally, the student-teacher ratio
is used as a proxy of the quality of the schooling system in a specific
country: it is assumed that the lower the number of students per
teacher, the greater the attention dedicated to each of them. We firstly
calculated the ratio at all the different educational levels, and then
we considered the average among them to attach a single value for each
country in the singular years.

We did not control for GDP per capita despite it is reasonable to assume
that wealthier countries will spend more in education as they have more
savings at their disposal, since we have already included this measure
in the independent variables' definition.

\hypertarget{categorical-variables}{%
\subsection{Categorical variables}\label{categorical-variables}}

We identified two main categorical variables. The first ones classify
countries by their geographical region---South America, North America,
Western Europe, Eastern Europe, Northern Europe, Middle East and
Oceania. We did not distinguish Southern Europe as the three countries
that formally belong to this group---Italy, Spain and Greece---could be
grouped in the previously defined categories. In this case, the base
category is Middle East.

Concerning the second group of categorical variables we decided to
classify countries by their level of income, according to the GDP per
capita. Since, by the classification of The World Bank (2021), they all
belong to the high-income cluster, we proceeded with a further division
by quartiles and to each of them we associated low-, lower middle-,
upper middle-, and upper-income categories. In this case, the base
category is represented by low-income countries.

\hypertarget{data-analysis}{%
\section{Data analysis}\label{data-analysis}}

\hypertarget{explanatory-data-analysis}{%
\subsection{Explanatory data analysis}\label{explanatory-data-analysis}}

In \emph{table X} a brief summary of our variables is presented, and
some key characteristics of the data's distribution are provided.
Starting with the dependent variable---unemployment---the average rate
among the 26 countries in the five years under focus is 8.49\%, with an
interesting minimum value of 2.8\% in Iceland in 2017, and a disastrous
maximum rate of 27.5\% in Greece in 2013, justified by the prolonged
crisis that affected the country. Looking at the independent variables,
it is relevant to highlight that the primary education is the schooling
level in which governments spend the highest percentage of GDP, compared
to other levels. The mean of the variable is 1.3\% and its maximum is
2.5\%; whilst for lower-secondary education they are 0.9\% and 1.3\%
respectively; for upper-secondary education they are 0.97\% and 1.63\%
respectively; for long-cycle tertiary they are 1.28\% and 2.05\%
respectively. Finally, we focus on the continuous control variables. The
mean of the average student-teacher ratio is 13.64, meaning that on
average, in the countries of our dataset, for each teacher---no matter
the educational level---follows 13 students. We must point out that the
average of student-teacher ratio is the lowest in lower-secondary
education (11.92) and the highest in long-cycle tertiary (15.8).
Concerning the educational attainment, the countries' mean is 34\%,
implying that on average, in the 26 countries of the sample, 34\% of the
population has concluded tertiary education. The minimum value is 16.3\%
in Italy in 2013, and the maximum is 50.9\% in Israel in 2017. If
associated with unemployment rates, this data highlights that in
countries where people are more educated, unemployment rates are lower
-- i.e., in Israel, to 50.9\% tertiary attainment is associated with an
unemployment rate of 4.2\% in the same year, whilst in Italy to 16.3\%
the unemployment was 12.1\% in the same year.

Finally, we made a summary of the categorical variables. Concerning the
distinction by income level, our 130 observations fall homogeneously in
the four categories. Indeed, we had 33,32,32,33 observations for lower,
lower-middle, upper-middle and upper income respectively. It is
important to highlight that, since we considered a period of five years,
some countries have experienced fluctuations of their income levels,
measured by GDP per capita. Therefore, in some cases, they changed their
categories from one year to another. For example, Italy was classified
as lower-middle in 2013, 2014, 2015, 2016 and changed to upper-middle in
2017. As regard the geographical categorization, we had 9,1,1,4,2,1,8
countries in Eastern Europe, Middle East, North America, Northern
Europe, Oceania, South America and Western Europe.

\emph{TABLE OF SUMMARY STATISTICS}

\hypertarget{modeling}{%
\subsection{Modeling}\label{modeling}}

Our 130 observations are arranged in form of panel data and we used
time-series cross-section-based estimation techniques.

Before regressing the models, we checked whether we needed a
fixed-effects or a random-effects model, performing the Hausman test. In
all the models summarized in \emph{TABLE X}, except for the third one,
the test showed a p-value lower 0.05, confirming the need of the fixed
effect. For model (3), we used random effect instead. This passage is
crucial to remove the effects of time-invariant characteristics and to
ensure that the models are truly measuring the impact of the independent
variables on the dependent one.\\
We started by running model (1) which comprised just the control
variables. It highlighted that the level of education attainment is
significant at a 99\% level (p-value = 7.47e-12), with an attractive
accuracy---R-squared of 41.16\%.

We proceeded by adding to the previous model the independent variables
one by one.\\
Concerning the models which tested the impact of expenditures in
primary, lower- and upper-secondary education on unemployment rates,
they all provided non-significant results (p-values \textgreater{}
0.10). Therefore, we could not reject the null hypothesis that stated
that devoting expenditures in these educational levels would not affect
unemployment. Thus, based on this data, our first three hypotheses were
rejected. Nonetheless, significant results were obtained in model (2),
which tried to linearly represents the relationship between expenditures
in long-cycle tertiary education and unemployment rates. Indeed, the
coefficient of the independent variable under focus appeared to be
significant at 95\% level (p-value = 0.02205) with an improved accuracy
with respect to the first model considered---R-squared of 46.28\%.

Given the results obtained, we decided to deepen our analysis and verify
whether the interaction between the ratio of long-cycle tertiary
education expenditure on GDP varied among different income levels and
geographical regions.

Starting with the distinction by economic conditions, model (3) did not
suggest any significant effect. Therefore, we could not reject the null
hypothesis which stated that there were no differences by income among
countries.

Moving to the division by regions, model (4) did not suggest any
significant effect and led to similar conclusions.

Although only the long-cycle tertiary education models are shown here,
models with these divisions were run for all educational level
expenditures, looking to see if there may be a significant impact on
unemployment in any of the levels according to income or to region. None
of the models were significant. Finally, it was important to verify
which of the three models better predicted the relationship we aimed to
study. We therefore performed the Wald-test to compare nested models.
For the comparison between the model with income categorization and the
one with only expenditures in long-cycle tertiary education, the test
gave us a p-value of 0.1192. Concerning the test between the model with
regional categorization and the one with only expenditures in long-cycle
tertiary education, we obtained a p-value of 0.6741. Given the results
obtained, we could conclude that, despite it including less variables,
the model with only one independent variable and no categorical ones is
a better predictor of the relationship we wanted to investigate.

\hypertarget{main-models}{%
\subsubsection{Main Models}\label{main-models}}

\begin{enumerate}
\def\labelenumi{\arabic{enumi})}
\item
  Model with controls
\item
  Model with controls + Long Cycle Tertiary Education Adjusted for GDPph
\item
  Model with controls + Long Cycle Tertiary Education Adjusted for GDPph
  + the Income Dummy
\item
  Model with controls + Long Cycle Tertiary Education Adjusted for GDPph
  + the Region Dummy
\end{enumerate}

\begin{Shaded}
\begin{Highlighting}[]
\FunctionTok{library}\NormalTok{(stargazer)}
\end{Highlighting}
\end{Shaded}

\begin{verbatim}
## 
## Please cite as:
\end{verbatim}

\begin{verbatim}
##  Hlavac, Marek (2018). stargazer: Well-Formatted Regression and Summary Statistics Tables.
\end{verbatim}

\begin{verbatim}
##  R package version 5.2.2. https://CRAN.R-project.org/package=stargazer
\end{verbatim}

\begin{Shaded}
\begin{Highlighting}[]
\FunctionTok{stargazer}\NormalTok{(modelControlF, modelLongCycleEducF, modelLongCycleTerEducIntIncomeALLR, modelLongCycleTerEducIntALLF, }\AttributeTok{type=}\StringTok{"text"}\NormalTok{, }\AttributeTok{header=}\ConstantTok{FALSE}\NormalTok{, }\AttributeTok{digits=}\DecValTok{2}\NormalTok{, }\AttributeTok{multicolumn =} \ConstantTok{FALSE}\NormalTok{, }\AttributeTok{out=}\StringTok{"./tables/summarymodelLongCycleEduc.html"}\NormalTok{, }\AttributeTok{title=}\StringTok{"Summary Statistics"}\NormalTok{,}
          \AttributeTok{notes =} \StringTok{"All the commands and algorithms are coded in R 4.0.3"}\NormalTok{)}
\end{Highlighting}
\end{Shaded}

\begin{verbatim}
## 
## Summary Statistics
## ================================================================================================================
##                                                                 Dependent variable:                             
##                                    -----------------------------------------------------------------------------
##                                        unemployment          unemployment      unemployment     unemployment    
##                                             (1)                   (2)              (3)              (4)         
## ----------------------------------------------------------------------------------------------------------------
## avRatio                                    0.002                 0.04             -0.005            0.07        
##                                           (0.11)                (0.11)            (0.11)           (0.12)       
##                                                                                                                 
## adultEducTer                             -0.57***              -0.51***          -0.39***         -0.52***      
##                                           (0.07)                (0.08)            (0.08)           (0.08)       
##                                                                                                                 
## longCycleTerEducGdp                                             2.41**             0.90             1.29        
##                                                                 (1.03)            (1.28)           (7.30)       
##                                                                                                                 
## lowerMiddle                                                                       -3.41                         
##                                                                                   (2.28)                        
##                                                                                                                 
## upperMiddle                                                                       -4.72                         
##                                                                                   (3.33)                        
##                                                                                                                 
## upper                                                                             -6.98                         
##                                                                                   (4.53)                        
##                                                                                                                 
## longCycleTerEducGdp:lowerMiddle                                                    1.92                         
##                                                                                   (1.68)                        
##                                                                                                                 
## longCycleTerEducGdp:upperMiddle                                                    1.59                         
##                                                                                   (2.62)                        
##                                                                                                                 
## longCycleTerEducGdp:upper                                                          3.24                         
##                                                                                   (3.37)                        
##                                                                                                                 
## longCycleTerEducGdp:southAmerica                                                                   -4.58        
##                                                                                                    (8.49)       
##                                                                                                                 
## longCycleTerEducGdp:westernEurope                                                                   2.07        
##                                                                                                    (9.52)       
##                                                                                                                 
## longCycleTerEducGdp:easternEurope                                                                   1.37        
##                                                                                                    (7.32)       
##                                                                                                                 
## longCycleTerEducGdp:northernEurope                                                                 -6.70        
##                                                                                                   (10.60)       
##                                                                                                                 
## longCycleTerEducGdp:oceania                                                                         4.76        
##                                                                                                    (9.18)       
##                                                                                                                 
## Constant                                                                         21.86***                       
##                                                                                   (3.82)                        
##                                                                                                                 
## ----------------------------------------------------------------------------------------------------------------
## Observations                                119                   112              112              112         
## R2                                         0.41                  0.46              0.42             0.48        
## Adjusted R2                                0.24                  0.30              0.37             0.28        
## F Statistic                        31.83*** (df = 2; 91) 24.41*** (df = 3; 85)   70.60***   9.35*** (df = 8; 80)
## ================================================================================================================
## Note:                                                                                *p<0.1; **p<0.05; ***p<0.01
##                                                             All the commands and algorithms are coded in R 4.0.3
\end{verbatim}

\hypertarget{validity}{%
\subsection{Validity}\label{validity}}

From the previous results, we concluded that the unique model which
provides us with significant outcomes is model (3). It is important to
test its validity, hence whether we collected the right data. To do so
we looked at three conditions: linearity, distribution and variability
of residuals.

Concerning the linearity conditions, plotting model (3) in a Cartesian
plane, except for some outliers, the relationship resembled a linear
one. To be more confident, we decided to test the quadratic effect.
However, the quadratic term was dropped, implying that its effect is
null. Hence the model should be linear, increasing its validity.

To test the distribution of the residuals, they have been plotted in a
histogram which represented symmetric residuals around zero. The
skewness test gave a g\_1 for our residuals smaller than the theoretical
threshold 2√6⁄n.~This confirmed our first impression from the graph,
that there is no skewness, strengthening model's validity.

The last condition to be verified was the residuals' homoskedasticity.
We applied the Breusch-Pagan test which gave us a p-value of 2.2e-16
which is much lower the 5\% significance level, implying
heteroskedasticity. Heteroskedasticity might deeply affect the validity
of our findings, however a way to obtain robust coefficients is through
the t-test on the latter. Since we obtained a significant beta for our
dependent variable in model (3), we were able to mitigate this
limitation. To conclude, the overall validity of the model was moderate:
the tests showed some confidence of linearity, no skewness but the
presence of heteroskedasticity.

\hypertarget{histogram-of-residuals}{%
\subsubsection{Histogram of Residuals}\label{histogram-of-residuals}}

\begin{Shaded}
\begin{Highlighting}[]
\FunctionTok{hist}\NormalTok{(modelLongCycleEducF}\SpecialCharTok{$}\NormalTok{residuals, }\AttributeTok{main =} \StringTok{"Histogram of Residuals"}\NormalTok{, }\AttributeTok{xlab=} \StringTok{"Residuals of Long Cycle Tertiary Education Model"}\NormalTok{)}
\end{Highlighting}
\end{Shaded}

\includegraphics{finalProjectTeam4_files/figure-latex/unnamed-chunk-3-1.pdf}

\hypertarget{discussion-of-results}{%
\section{Discussion of results}\label{discussion-of-results}}

From the data analysis just presented it can be concluded that none of
our initial hypotheses can be accepted. Indeed, data do not show any
significant relationship between governmental spending in primary,
lower- and upper-secondary education and unemployment rate. They only
highlight a significant and---opposite from what expected---positive
relationship between long-cycle tertiary education expenditures and
unemployment rates. The results might seem discouraging initially, but
we had the opportunity to gain important insights. First, despite the
imperative to provide and spend in lower grades of schooling, the
effects of these investments are not directly captured by unemployment
levels. Following, the positive relationship between unemployment rates
and outlays in long-cycle tertiary education might appear
counter-intuitive. However, it tells us that countries which spend more
in higher education for more educated individuals who put upward
pressure to labour supply. If the latter is not matched by a sufficient
labour demand, it leads to an increase in unemployment rates, which is
the exact mechanism experienced in the countries of our sample.

\hypertarget{conclusions}{%
\section{Conclusions}\label{conclusions}}

Our study shows some limitations. Firstly, data were not available for
all the OECD countries and, therefore, we were forced to restrict our
analysis to just 26 of them and for the period between 2013-2017. This
led us to further limitations when considering possible division among
countries, all of them being categorized as high-income by the World
Bank. Moreover, our study being one of the firsts to focus on
governmental expenditures broken down by different schooling levels, we
decided not to distinguish unemployment among different ages but to
consider general levels for the workers aged 25-64. Nonetheless, it is
too broad of a group, and it is likely subjected to other factors apart
from government spending in education.

Hence, for further research, we suggest broadening the time frame, and
considering a greater sample of countries---from various regions and
with different levels of income---to increase the generalizability of
the study. Secondly, future analysis should consider as a dependent
variable a smaller percentage of the working population, preferably
focusing on younger individuals to better assess the real impact of
expenditures on their hiring possibilities. Lastly, different types of
relationships could be investigated -- i.e., the logarithmic form.

On the basis of these considerations, we thoroughly described all the
steps made to make our study reproducible and so to be used as a
reference for future studies in the field.

\hypertarget{bibliography}{%
\section{Bibliography}\label{bibliography}}

Agasisti, T. (2014). The Efficiency of Public Spending on Education: an
Empirical Comparison of EU countries. \emph{European Journal of
Education}, 49(4).

Ahec Šonje, A., Deskar-Škrbić, M., \& Šonje, V. (2018). \emph{Efficiency
of Public Expenditure on Education: Comparing Croatia with other NMS}.
MPRA.

Bai, L. (2006). Graduate Unemployment: Dilemmas and Challenges in
China's Move to Mass Higher Education. \emph{The China Quarterly},
185(128-144).

Baqir, M., Harun, M., \& Razani, M. J. (2017). Employment Generated by
Government Spending on Education. \emph{International Journal of
Academic Research in Business and Social Sciences}, 7(2).

Chandra, A. (2010). \emph{Does Government Expenditure on Education
Promote Economic Growth? An Econometric Analysis}. MPRA Working Paper.

Dachito, A. C., Alemu, M., \& Alemu, B. (2020). The Impact of Public
Education Expenditures on Graduate Unemployment: Cointegration Analysis
to Ethiopia. \emph{Journal of International Trade, Logistics and Law},
6(2).

Fedderke, J., \& Luiz, J. (2006). Infrastructure Investment in Long Run
Economic Growth: South Africa 1875-2001. \emph{World Development},
34(6), 1037-1059.

Flores, I. (2017). Modelling Efficiency in Education: How are European
Countries Spending their Budgets and What Relation Between Money and
Performance. \emph{Sociologia}, 83.

Gimenez, V., Prior, D., \& Thieme, C. (2007). Technical Efficiency,
Managerial Efficiency and Objective Setting in the Educational System:
an International Comparison. \emph{Journal of Operational Research
Society}, 58(8), 996-1007.

Grimaccia, E., \& Lima, R. (2013). \emph{Public Expenditure on
Education, Education Attainment and Employment: a Comparison among
European Countries}. Istat, Italy.

Jorgenson, D. W., \& Fraumeni, B. M. (1992). Investmeent in Education
and U.S. Economic Growth. \emph{The Scandinavian Journal of Economics},
94, 51-70.

Jung, H., \& Thorbecke, E. (2001). \emph{The Impact of Public Education
Expenditure on Human Capital, Growth and Poverty in Tanzania and Zambia:
A General Equilibrium Approach}. International Monetary Fund.

Mefteh, H., Bouhajeb, M., \& Smaoui, F. (2016). \emph{Higher education,
Graduate unemployment, Poverty and Economic growth in Tunisia,
1990-2013}. Economic Analysis Working Papers (2002-2010). Atlantic
Review of Economics (2011-2016).

Nurudeen, A., \& Usman, A. (2010). Government Expenditure and Economic
Growth in Nigeria, 1970-2008: A Disaggregated Analysis. \emph{Business
and Economics Journal}, 4, 1-11.

Obi, Z. C., \& Obi, C. O. (2014). Impact of Government Expenditure on
Education: the Nigerian Experience. \emph{International Journal of
Business and Finance Management Research}, 42-48.

OECD. (2019). \emph{Education at a Glance 2019: OECD Indicators}. Paris:
OECD Publishing.

OECD. (2020). \emph{Education at a Glance 2020: OECD Indicators}. OECD.
Paris: OECD Publishing. Tratto da OECD.

Quan, N. T., \& Beck, J. H. (1987). Public Education Expenditures and
State Economic Growth: Northeast and Sunbelt Regions. \emph{Southern
Economic Journal}, 54(2), 361-376.

Zafer, P., William, A. O., \& Leslie, S. K. (2014). The Long-term Impact
of Educational and Health Spending on Unemployment Rates. \emph{European
Journal of Economic and Political Studies}, 7(1), 49-69.

\hypertarget{appendices}{%
\section{Appendices}\label{appendices}}

\hypertarget{paths-to-be-followed-to-retrieve-data}{%
\subsection{Paths to be followed to retrieve
data}\label{paths-to-be-followed-to-retrieve-data}}

\hypertarget{independent-variable}{%
\subsubsection{Independent variable:}\label{independent-variable}}

\url{https://stats.oecd.org/Index.aspx?DataSetCode=EAG_FIN_SOURCE}

Filters: countries in sample / years: 2013-2017 / ISC11: Primary
education (ISCED2011 level 1) + Lower secondary education (ISCED2011
level 2) + Upper secondary education (ISCED2011 level 3) + Long cycle
tertiary (ISCED2011 levels 6 to 8) / REF\_SECTOR: all sectors /
COUNTERPART\_SECTOR: all public and private institutions /
EXPENDITURE\_TYPE: all expenditure types / UNIT\_ MEASURE: USD
Purchasing Power Parity

\hypertarget{control-variables-1}{%
\subsubsection{Control variables:}\label{control-variables-1}}

GDP per head: \url{https://stats.oecd.org/Index.aspx?QueryId=54369}
Filters: countries in sample / years: 2013-2017 / Measure: USD constant
prices, 2015 PPPs Path: Gross domestic product (GDP) : GDP per head, US
\$, constant prices, constant PPPs, reference year 2015

GDP: \url{https://stats.oecd.org/Index.aspx?QueryId=54369} Path: Gross
domestic product (GDP) : GDP, US \$, current prices, current PPPs,
millions Filters: countries in sample / years: 2013-2017 /

Educational attainment:
\url{https://data.oecd.org/eduatt/adult-education-level.htm\#indicator-chart}
Path: Adult Education Level Filters: countries in sample / Perspective:
Tertiary / time: years: 2013-2017

Student-teacher ratio: \url{https://stats.oecd.org/}\\
Path: Education and training -\textgreater{} education at a glance
-\textgreater{} teachers and the learning environment -\textgreater{}
student-teacher ration and average class size.

\hypertarget{dependent-variable-1}{%
\subsubsection{Dependent variable:}\label{dependent-variable-1}}

\url{https://stats.oecd.org/}~\\
Path: Education and training -\textgreater{} education at a glance
-\textgreater{} educational attainment and outcomes-\textgreater{}
educational attainment and labor force status-\textgreater{} Employment,
unemployment and inactivity rate of 25-64 year-olds, by programme
orientation Filters: indicator: unemployment /

\end{document}
